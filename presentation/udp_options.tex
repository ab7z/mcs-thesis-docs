\documentclass[aspectratio=169]{beamer}
\usepackage[utf8]{inputenc}
\usepackage[T1]{fontenc}
\usepackage[ngerman]{babel}
\usepackage{tikz}
\usetikzlibrary{positioning,arrows.meta,calc,decorations.pathreplacing}
\usepackage{pgfgantt}

% Farben: nur weiß, schwarz, blau
\definecolor{mainblue}{RGB}{0,51,102}

% Beamer-Theme ohne Boxen
\usetheme{default}
\usecolortheme{default}

\setbeamercolor{frametitle}{fg=mainblue}
\setbeamercolor{title}{fg=mainblue}
\setbeamercolor{structure}{fg=mainblue}
\setbeamercolor{itemize item}{fg=mainblue}
\setbeamercolor{itemize subitem}{fg=black}
\setbeamertemplate{itemize item}{\textbullet}
\setbeamertemplate{itemize subitem}{--}
\setbeamertemplate{navigation symbols}{}
\setbeamertemplate{footline}[frame number]

\title{Implementierung von UDP Options}
\subtitle{RFC 9868 in Zig mit eBPF-Monitoring}
\author{Alan Bernstein}
\date{}
\institute{FernUniversität in Hagen}

\begin{document}

% Folie 1: Titel
\begin{frame}
\titlepage
\end{frame}

% Folie 2: Problemstellung
\begin{frame}{Problemstellung: UDP heute}
\begin{columns}[T]
\begin{column}{0.55\textwidth}
\textbf{Limitierungen von UDP (RFC 768, 1980):}
\begin{itemize}
    \item Keine native Integritätsprüfung der Payload
    \item Fehlende Authentifizierung auf Transportebene
    \item Keine Path MTU Discovery
    \item Keine Timestamps für RTT-Messung
\end{itemize}

\vspace{1em}
\textit{Minimalistisch, aber unflexibel für moderne Anforderungen.}
\end{column}
\begin{column}{0.4\textwidth}
\centering
\textbf{UDP Header (8 Bytes)}

\vspace{0.5em}
\begin{tikzpicture}[scale=0.8]
    \draw[thick, mainblue] (0,0) rectangle (2.5,0.8);
    \draw[thick, mainblue] (2.5,0) rectangle (5,0.8);
    \draw[thick, mainblue] (0,-0.8) rectangle (2.5,0);
    \draw[thick, mainblue] (2.5,-0.8) rectangle (5,0);

    \node at (1.25,0.4) {\small Source Port};
    \node at (3.75,0.4) {\small Dest Port};
    \node at (1.25,-0.4) {\small Length};
    \node at (3.75,-0.4) {\small Checksum};

    \node[below] at (1.25,-0.8) {\footnotesize 16 bits};
    \node[below] at (3.75,-0.8) {\footnotesize 16 bits};
\end{tikzpicture}

\vspace{0.5em}
{\footnotesize Seit 45 Jahren unverändert}
\end{column}
\end{columns}
\end{frame}

% Folie 3: Warum UDP Options?
\begin{frame}{Warum werden UDP Options benötigt?}
\textbf{Drei zentrale Gründe:}

\begin{itemize}
    \item \textbf{Legacy-Kompatibilität} \\
    Bestehende UDP-Anwendungen können nicht einfach auf DTLS migriert werden. \\
    Options werden von Legacy-Empfängern ignoriert.

    \item \textbf{Performance-Anforderungen} \\
    Echtzeit-Anwendungen benötigen minimale Latenz. \\
    UDP Options arbeiten direkt auf Transportebene -- ohne zusätzliche Schicht.

    \item \textbf{Middlebox-Traversal} \\
    NATs und Firewalls müssen UDP-Pakete korrekt weiterleiten. \\
    RFC 9868 definiert dafür die Options Checksum (OCS).
\end{itemize}

\vspace{1em}
\textbf{Prominente UDP-basierte Protokolle:} \quad
QUIC (HTTP/3) \quad | \quad WebRTC \quad | \quad DNS
\end{frame}

% Folie 4: RFC 9868 und Surplus Area
\begin{frame}{RFC 9868: Transport Options for UDP}
\begin{columns}[T]
\begin{column}{0.5\textwidth}
\textbf{Definition (Oktober 2025):}
\begin{itemize}
    \item Erste Erweiterung von RFC 768 nach 45 Jahren
    \item Options im \textit{Surplus Area} zwischen UDP-Payload und IP-Ende
    \item Abwärtskompatibel zu Legacy-Empfängern
\end{itemize}

\vspace{1em}
\textbf{Kernprinzipien:}
\begin{itemize}
    \item UDP bleibt zustandslos
    \item UDP bleibt unidirektional
    \item Options sind ein Framework
\end{itemize}
\end{column}
\begin{column}{0.45\textwidth}
\centering
\textbf{UDP Datagram mit Options}

\vspace{0.5em}
\begin{tikzpicture}[scale=0.75]
    % IP Header
    \draw[thick, black] (0,4) rectangle (5,4.6);
    \node at (2.5,4.3) {\small IP Header};

    % UDP Header
    \draw[thick, mainblue, fill=mainblue!20] (0,3) rectangle (5,4);
    \node at (2.5,3.5) {\small UDP Header (8 Bytes)};

    % User Data
    \draw[thick, black] (0,1.5) rectangle (5,3);
    \node at (2.5,2.25) {\small User Data (Payload)};
    \node[right] at (5.1,2.25) {\footnotesize $\leftarrow$ Length-Feld};

    % Surplus Area
    \draw[thick, mainblue, fill=mainblue!10] (0,0.5) rectangle (5,1.5);
    \node[mainblue] at (2.5,1) {\textbf{Surplus Area}};
    \node[mainblue, below] at (2.5,0.5) {\footnotesize UDP Options};

    % Brace
    \draw[decorate, decoration={brace, amplitude=5pt, mirror}, thick, mainblue]
        (5.3,0.5) -- (5.3,1.5);
    \node[right, mainblue] at (5.5,1) {\footnotesize neu};
\end{tikzpicture}
\end{column}
\end{columns}
\end{frame}

% Folie 5: UDP Options Architektur
\begin{frame}{UDP Options Architektur}
\begin{columns}[T]
\begin{column}{0.55\textwidth}
\textbf{UDP Options Framework:}

\begin{itemize}
    \item Erweiterbare Architektur für zukünftige Anforderungen
    \item Options werden im TLV-Format (Type-Length-Value) kodiert
    \item Kategorisierung in SAFE und UNSAFE Options
    \item Unterstützung für Sicherheit, Fragmentierung und Timing
\end{itemize}

\vspace{0.5em}
{\small Definierte Typen: OCS, FRAG, TIME, AUTH, EOL, NOP}
\end{column}
\begin{column}{0.4\textwidth}
\centering
\textbf{Option Format (TLV)}

\vspace{1em}
\begin{tikzpicture}[scale=0.8]
    % Kind
    \draw[thick, mainblue, fill=mainblue!30] (0,0) rectangle (1.5,0.8);
    \node at (0.75,0.4) {\small Kind};
    \node[below] at (0.75,0) {\footnotesize 8 bits};

    % Length
    \draw[thick, mainblue, fill=mainblue!20] (1.5,0) rectangle (3.0,0.8);
    \node at (2.25,0.4) {\small Length};
    \node[below] at (2.25,0) {\footnotesize 8 bits};

    % Data
    \draw[thick, mainblue, fill=mainblue!10] (3.0,0) rectangle (7.0,0.8);
    \node at (5.0,0.4) {\small Data};
    \node[below] at (5.0,0) {\footnotesize 0--255 Bytes};
\end{tikzpicture}

\vspace{1.5em}
{\small Type-Length-Value Schema}
\end{column}
\end{columns}
\end{frame}

% Folie 6: Anwendungsfälle
\begin{frame}{Anwendungsfälle für UDP Options}
\begin{columns}[T]
\begin{column}{0.48\textwidth}
\textbf{Sicherheit \& Integrität:}
\begin{itemize}
    \item Authentifizierung und Integritätsschutz (AUTH)
    \item Path MTU Discovery ohne Payload-Beeinflussung
\end{itemize}
\end{column}
\begin{column}{0.48\textwidth}
\textbf{Echtzeit \& Performance:}
\begin{itemize}
    \item Präzise RTT-Messung mit Timestamps (TIME)
    \item UDP-Level Fragmentierung statt IP (FRAG)
\end{itemize}
\end{column}
\end{columns}

\vspace{2em}
\centering
\textbf{Einsatzgebiete:} \quad VoIP \quad | \quad IoT \quad | \quad Gaming \quad | \quad Streaming
\end{frame}

% Folie 7: Zusammenfassung
\begin{frame}{Zusammenfassung}
\textbf{RFC 9868 erweitert UDP um:}
\begin{itemize}
    \item Sicherheitsfunktionen (Authentifizierung, Integrität)
    \item Path MTU Discovery
    \item Timestamps für RTT-Messung
    \item Fragmentierung auf UDP-Ebene
\end{itemize}

\vspace{1em}
\textbf{Vorteile:}
\begin{itemize}
    \item Abwärtskompatibel -- Legacy-Empfänger ignorieren Options
    \item Kein zusätzlicher Protokoll-Overhead wie bei DTLS
    \item Erweiterbar durch neue Option-Typen
\end{itemize}

\end{frame}

% Folie 8: Zeitplan
\begin{frame}{Zeitplan}
\centering
\begin{ganttchart}[
    x unit=1.6cm,
    y unit chart=0.6cm,
    vgrid,
    hgrid,
    bar/.style={fill=mainblue},
    bar height=0.5,
    title height=1,
    title label font=\small,
    bar label font=\small,
    group label font=\small\bfseries
]{1}{6}
    \gantttitle{Monat}{6} \\
    \gantttitlelist{1,...,6}{1} \\
    \ganttbar{Literaturrecherche}{1}{1} \\
    \ganttbar{Architekturentwurf}{2}{2} \\
    \ganttbar{Implementierung (Zig)}{2}{4} \\
    \ganttbar{eBPF Monitoring}{4}{5} \\
    \ganttbar{Evaluation}{5}{6} \\
    \ganttbar{Dokumentation}{1}{6}
\end{ganttchart}
\end{frame}

\end{document}
