%!TEX encoding = UTF-8 Unicode
%!TEX program = pdflatex
% =============================================================================
% Masterarbeit: UDP Transport Options (RFC 9868)
%               Implementierung und Analyse in Zig
% =============================================================================
% Autor: Alan Bernstein
% Matrikelnummer: 2068834
% FernUniversität in Hagen - LG Kooperative Systeme
% =============================================================================

\documentclass[
    12pt,           % Schriftgröße
    a4paper,        % Papierformat
    oneside,        % Einseitiger Druck
    openany         % Kapitel auf jeder Seite beginnen (keine leeren Seiten)
]{report}

% =============================================================================
% 1. KONFIGURATION (ehemals config/preamble.tex)
% =============================================================================

% Encoding und Sprache
\usepackage[utf8]{inputenc}
\usepackage[T1]{fontenc}
\usepackage[ngerman]{babel}
\usepackage{csquotes}           % Korrekte Anführungszeichen

% Schrift und Layout
\usepackage{lmodern}            % Latin Modern Schrift
\usepackage{microtype}          % Bessere Typographie
\usepackage[
    left=3cm,
    right=3cm,
    top=2.5cm,
    bottom=2.5cm
]{geometry}

% Zeilenabstand 1,5-fach
\usepackage{setspace}
\onehalfspacing

% Kompakte Kapitelüberschriften (ohne titlesec - verursacht leere Seiten)
% Kapitelformat direkt über LaTeX-Befehle anpassen
\makeatletter
\renewcommand{\@makechapterhead}[1]{%
  \vspace*{-20pt}%
  {\parindent \z@ \raggedright \normalfont
    \ifnum \c@secnumdepth >\m@ne
        \LARGE\bfseries \thechapter.\hspace{1em}%
    \fi
    \interlinepenalty\@M
    \LARGE \bfseries #1\par\nobreak
    \vskip 30\p@
  }}
\renewcommand{\@makeschapterhead}[1]{%
  \vspace*{-20pt}%
  {\parindent \z@ \raggedright \normalfont
    \interlinepenalty\@M
    \LARGE \bfseries #1\par\nobreak
    \vskip 30\p@
  }}
% Erlaube Seitenumbrüche zwischen aufeinanderfolgenden Überschriften ohne Fließtext.
% Ohne diesen Patch bilden Überschriften eine unbrechbare Kette (@nobreaktrue kaskadiert),
% die bei Stub-Kapiteln die Seitenhöhe übersteigt und leere Seiten erzeugt.
\let\@orig@startsection\@startsection
\def\@startsection{\@nobreakfalse\@orig@startsection}
\makeatother

% Abstand zwischen Fußnotennummer und -text
\makeatletter
\renewcommand{\@makefntext}[1]{%
  \noindent\@makefnmark\enspace #1}
\makeatother

% Seitenstil: leere Kopfzeile, Seitennummer zentriert im Fuß
\usepackage{fancyhdr}
\fancypagestyle{plain}{
    \fancyhf{}
    \fancyfoot[C]{\thepage}
    \renewcommand{\headrulewidth}{0pt}
}
\pagestyle{plain}

% Bibliographie
\usepackage[
    backend=biber,
    style=numeric-comp,
    sorting=none,
    giveninits=true,
    maxbibnames=99
]{biblatex}
\addbibresource{literatur.bib}

% Grafiken und Tabellen
\usepackage{graphicx}
\graphicspath{{figures/}}
\usepackage{float}
\usepackage{booktabs}           % Schöne Tabellen
\usepackage{tabularx}           % Flexible Tabellen
\usepackage{longtable}          % Mehrseitige Tabellen
\usepackage{multirow}           % Mehrzeilige Zellen
\usepackage{amssymb}            % Für \checkmark Symbol

% Code-Listings
\usepackage{listings}
\usepackage{fancyvrb}            % Zentrierbare Verbatim-Umgebungen
\usepackage{xcolor}

% Farben für Listings
\definecolor{codegreen}{rgb}{0,0.6,0}
\definecolor{codegray}{rgb}{0.5,0.5,0.5}
\definecolor{codepurple}{rgb}{0.58,0,0.82}
\definecolor{backcolour}{rgb}{0.97,0.97,0.97}

\lstdefinestyle{codestyle}{
    backgroundcolor=\color{backcolour},
    commentstyle=\color{codegreen},
    keywordstyle=\color{blue},
    numberstyle=\tiny\color{codegray},
    stringstyle=\color{codepurple},
    basicstyle=\ttfamily\footnotesize,
    breakatwhitespace=false,
    breaklines=true,
    captionpos=b,
    keepspaces=true,
    numbers=left,
    numbersep=5pt,
    showspaces=false,
    showstringspaces=false,
    showtabs=false,
    tabsize=2,
    frame=single
}
\lstset{style=codestyle}

% Sprachdefinitionen für Code
\lstdefinelanguage{Zig}{
    keywords={const, var, fn, pub, return, if, else, while, for, switch, break, continue,
              try, catch, defer, errdefer, orelse, and, or, not, struct, enum, union,
              packed, extern, export, inline, comptime, test, unreachable, undefined,
              null, true, false},
    keywordstyle=\color{blue}\bfseries,
    ndkeywords={u8, u16, u32, u64, i8, i16, i32, i64, usize, isize, bool, void, anytype, type,
                @intCast, @truncate, @ptrCast, @alignCast, @bitCast, @import, @This, @typeName},
    ndkeywordstyle=\color{codepurple}\bfseries,
    sensitive=true,
    comment=[l]{//},
    morecomment=[s]{/*}{*/},
    morestring=[b]"
}

\lstdefinelanguage{Hexdump}{
    keywords={},
    sensitive=false,
    morecomment=[l]{|}
}

% Hyperlinks
\usepackage{hyperref}
\hypersetup{
    colorlinks=true,
    linkcolor=black,
    citecolor=black,
    filecolor=black,
    urlcolor=blue,
    pdftitle={Masterarbeit - UDP Transport Options (RFC 9868)},
    pdfauthor={Alan Bernstein}
}

% Abkürzungsverzeichnis
\usepackage[
    acronym,
    toc,
    nonumberlist
]{glossaries}
\makeglossaries

% Abkürzungen definieren
% RFC 9868 Options
\newacronym{ocs}{OCS}{Option Checksum}
\newacronym{frag}{FRAG}{Fragmentation Option}
\newacronym{time}{TIME}{Timestamp Option}
\newacronym{auth}{AUTH}{Authentication Option}
\newacronym{eol}{EOL}{End of List}
\newacronym{nop}{NOP}{No Operation}
\newacronym{apc}{APC}{Alternate Payload Checksum}
\newacronym{mds}{MDS}{Maximum Datagram Size}
\newacronym{mrds}{MRDS}{Maximum Reassembled Datagram Size}

% Verwandte Technologien
\newacronym{dtls}{DTLS}{Datagram Transport Layer Security}
\newacronym{pmtu}{PMTU}{Path Maximum Transmission Unit}

% Sonstiges
\usepackage{enumitem}           % Anpassbare Listen
\usepackage{etoolbox}           % Zum Patchen von Befehlen
\usepackage{caption}            % Bildunterschriften anpassen
\usepackage{subcaption}         % Subfigures
\usepackage{pdfpages}           % PDF-Seiten einbinden
\usepackage[disable]{todonotes} % TODO-Notizen deaktiviert

% =============================================================================
% 2. EIGENE BEFEHLE (ehemals config/commands.tex)
% =============================================================================

% Textauszeichnung
\newcommand{\term}[1]{\textit{#1}}
\newcommand{\code}[1]{\texttt{#1}}
\newcommand{\file}[1]{\texttt{#1}}
\newcommand{\filepath}[1]{\texttt{#1}}
\newcommand{\env}[1]{\texttt{\$#1}}

% Referenzen
\newcommand{\chapref}[1]{Kapitel~\ref{#1}}
\newcommand{\secref}[1]{Abschnitt~\ref{#1}}
\newcommand{\figref}[1]{Abbildung~\ref{#1}}
\newcommand{\tabref}[1]{Tabelle~\ref{#1}}
\newcommand{\lstref}[1]{Listing~\ref{#1}}

% Fachbegriffe für UDP Options
\newcommand{\surplusarea}{\term{Surplus Area}}
\newcommand{\optionchecksum}{\term{Option Checksum (OCS)}}
\newcommand{\fragoption}{\term{FRAG Option}}
\newcommand{\timeoption}{\term{TIME Option}}
\newcommand{\authoption}{\term{AUTH Option}}
\newcommand{\rfcref}[1]{RFC~#1}
\newcommand{\rfcnine}{\rfcref{9868}}

% TODO-Markierungen
\newcommand{\todoref}[1]{\todo[color=yellow!40]{Referenz: #1}}
\newcommand{\todocontent}[1]{\todo[color=red!40]{Inhalt: #1}}
\newcommand{\todoreview}[1]{\todo[color=green!40]{Review: #1}}

% Zitate
\newcommand{\zitat}[1]{\glqq#1\grqq}


% =============================================================================
% 3. METADATEN (ehemals config/metadata.tex)
% =============================================================================

\newcommand{\thesistitle}{UDP Transport Options}
\newcommand{\thesissubtitle}{Implementierung und Analyse von RFC 9868 in Zig}
\newcommand{\thesisauthor}{Alan Bernstein}
\newcommand{\thesismatrikel}{2068834}
\newcommand{\thesisemail}{alan.bernstein@pm.me}

\newcommand{\thesissupervisor}{Prof. Dr. Christian Icking}
\newcommand{\thesissecondexaminer}{Dr. Lihong Ma}

\newcommand{\thesisuniversity}{FernUniversität in Hagen}
\newcommand{\thesisfaculty}{Fakultät für Mathematik und Informatik}
\newcommand{\thesisdepartment}{LG Kooperative Systeme}

\newcommand{\thesisdegree}{Master of Science (M.Sc.)}
\newcommand{\thesisprogram}{Master Informatik}

\newcommand{\thesisdate}{\today}
\newcommand{\thesissubmissiondate}{[Datum der Abgabe]}


% =============================================================================
% DOKUMENT
% =============================================================================
\begin{document}

% -----------------------------------------------------------------------------
% FRONTMATTER
% -----------------------------------------------------------------------------
\pagenumbering{roman}
\pagestyle{plain}

% --- TITELSEITE ---
\begin{titlepage}
    \centering
    \vspace*{1cm}
    {\large \thesisuniversity}\\[0.3cm]
    {\large \thesisfaculty}\\[0.3cm]
    {\large \thesisdepartment}\\[2cm]
    {\LARGE \bfseries \thesistitle}\\[0.5cm]
    {\large \thesissubtitle}\\[1.5cm]
    {\Large Masterarbeit}\\[0.5cm]
    {\large im Studiengang \thesisprogram}\\[2cm]
    {\large vorgelegt von}\\[0.3cm]
    {\Large \thesisauthor}\\[0.3cm]
    {\normalsize Matrikelnummer: \thesismatrikel}
    \vfill
    \begin{tabular}{ll}
        \textbf{Erstprüfer:} & \thesissupervisor \\[0.3cm]
        \textbf{Zweitprüferin:} & \thesissecondexaminer \\[0.3cm]
    \end{tabular}
    \vfill
    {\large Hagen, \thesissubmissiondate}
\end{titlepage}

% --- VERZEICHNISSE ---
\tableofcontents

\listoffigures
\addcontentsline{toc}{chapter}{Abbildungsverzeichnis}

\listoftables
\addcontentsline{toc}{chapter}{Tabellenverzeichnis}

% Abkürzungsverzeichnis
\glsaddall
\printglossary[type=\acronymtype, title=Abkürzungsverzeichnis]

% -----------------------------------------------------------------------------
% MAINMATTER (Inhalt)
% -----------------------------------------------------------------------------
\pagenumbering{arabic}
\pagestyle{plain}

% =============================================================================
% Kapitel 1: Einleitung
% =============================================================================

\chapter{Einleitung}
\label{chap:einleitung}

\section{Motivation}
\label{sec:motivation}

Das User Datagram Protocol (UDP) wurde 1980 in RFC~768 spezifiziert und ist seitdem praktisch unverändert
geblieben \cite{rfc768}. Als minimalistisches Transportprotokoll bietet UDP lediglich Portnummern zur
Demultiplexierung und eine optionale Prüfsumme zur Fehlererkennung. Diese bewusste Einfachheit macht UDP zu einem
der am häufigsten genutzten Transportprotokolle im Internet.

Moderne Anwendungen nutzen UDP zunehmend für zeitkritische Kommunikation. Protokolle wie QUIC, WebRTC, DNS und
zahlreiche Spiele- sowie IoT-Anwendungen setzen auf UDP, um die Latenz des TCP-Handshakes zu vermeiden und mehr
Kontrolle über die Übertragungsmechanismen zu erhalten. Dabei müssen Funktionen, die TCP auf Transportebene
bereitstellt -- wie Fragmentierung, Zeitstempel oder erweiterte Prüfsummen -- in der Anwendungsschicht
reimplementiert werden.

RFC~9868, veröffentlicht im Oktober 2025, schließt diese Lücke durch die Einführung von \emph{Transport Options
for UDP} \cite{rfc9868}. Die Erweiterung nutzt die sogenannte \emph{Surplus Area} -- den Bereich zwischen dem Ende
der UDP-Nutzdaten und dem Ende des IP-Datagramms -- für optionale Transportschicht-Mechanismen. Da das
UDP-Length-Feld die Länge der Nutzdaten angibt und diese von der IP-Paketlänge abweichen kann, entsteht ein bisher
ungenutzter Bereich für Erweiterungen.

RFC~9868 definiert ein erweiterbares Options-Framework mit folgenden Eigenschaften:

\begin{itemize}
    \item Optionen werden im TLV-Format (Type-Length-Value) kodiert und in der \emph{Surplus Area} platziert
    \item Kategorisierung in SAFE Options (können ignoriert werden) und UNSAFE Options (modifizieren Nutzdaten)
    \item Unterstützung für Sicherheitsfunktionen (Prüfsummen, Integrität)
    \item Fragmentierung auf UDP-Ebene, unabhängig von IP-Fragmentierung
\end{itemize}

Ein entscheidender Vorteil der UDP-Options ist ihre \textbf{Abwärtskompatibilität}: Legacy-Systeme, die UDP Options
nicht unterstützen, ignorieren die Surplus Area und verarbeiten lediglich die regulären UDP-Nutzdaten. Dies
ermöglicht eine schrittweise Einführung der Erweiterung ohne Beeinträchtigung bestehender Infrastruktur.


\section{Problemstellung}
\label{sec:problemstellung}

Die Implementierung von UDP Options gemäß RFC~9868 wirft mehrere technische Herausforderungen auf, die in dieser
Arbeit adressiert werden.

\subsection{Das Surplus-Area-Konzept}

Anders als TCP, dessen Header ein explizites Optionsfeld vorsieht, bietet der ursprüngliche UDP-Header keinen Raum
für Erweiterungen. RFC~9868 löst dieses Problem durch Nutzung der \emph{Surplus Area}: Da das UDP-Length-Feld die
Länge des UDP-Headers plus Nutzdaten angibt, kann diese kleiner sein als die vom IP-Header implizierte
Transportnutzlast. Der entstehende Bereich wird für UDP Options verwendet:

\begin{verbatim}
                       IP transport payload
    <---------------------------------------------------------->
   +--------+---------+----------------------+------------------+
   | IP Hdr | UDP Hdr | UDP user data        | surplus area     |
   +--------+---------+----------------------+------------------+
             <------------------------------>
                        UDP Length
\end{verbatim}

Die Surplus Area beginnt mit einem 2-Byte Option Checksum (OCS), gefolgt von den eigentlichen Optionen im
TLV-Format (Type-Length-Value). Diese Struktur ermöglicht variable Optionslängen und zukünftige Erweiterungen.

\subsection{Unidirektionalität und Statelessness}

RFC~9868 stellt explizit klar: \emph{``UDP is unidirectional; UDP Options do not change that fact.''}
\cite[Sec.~6]{rfc9868}. Diese Design-Entscheidung hat weitreichende Implikationen:

\begin{itemize}
    \item \textbf{Keine Option erfordert eine Antwort:} Keine der in RFC~9868 definierten Optionen verpflichtet den
    Empfänger zu einer Reaktion. Selbst REQ (Echo Request) erzeugt keine automatische Antwort auf Protokollebene.

    \item \textbf{Antworten sind Anwendungslogik:} Wenn ein Empfänger auf eine REQ-Option mit einer RES-Option
    (Echo Response) antwortet, geschieht dies durch eine bewusste Entscheidung der Anwendung, nicht durch einen
    Protokollmechanismus. Die Anwendung entscheidet, \emph{wann} und \emph{ob} eine Antwort gesendet wird.
\end{itemize}

Für Anwendungsfälle, die bidirektionale Kommunikation oder garantierte Antworten erfordern, verweist RFC~9868 auf
ein separates Dokument, das solche Mechanismen spezifizieren soll. Dies unterstreicht, dass UDP Options bewusst auf
der unidirektionalen Natur von UDP aufbauen.

Für die Implementierung bedeutet dies, dass beide Kommunikationspartner als symmetrische Peers agieren: Jeder Peer
muss sowohl senden als auch empfangen können, ohne dass das Protokoll eine Client-Server-Hierarchie vorschreibt.

\subsection{Implementierungsherausforderungen}

Die zentrale Herausforderung bei der Implementierung von UDP Options liegt im Zugriff auf die Surplus Area.
Existierende Betriebssystem-Stacks liefern über Standard-Socket-APIs ausschließlich die durch das UDP-Length-Feld
begrenzten Nutzdaten an die Anwendungsschicht; die Surplus Area wird stillschweigend verworfen
\cite[Sec.~18]{rfc9868}. Da die ursprüngliche UDP-Spezifikation \cite{rfc768} keinen Mechanismus für den Zugriff
auf Daten jenseits der UDP-Länge vorsieht, erfordert eine Implementierung im Userspace den Einsatz von Raw Sockets,
die vollständige Kontrolle über die Paketstruktur ermöglichen, jedoch erhöhte Systemberechtigungen voraussetzen.

Eine weitere Herausforderung stellt die Kompatibilität mit Middleboxes dar. Während NAT-Geräte UDP-Pakete mit
Surplus Area typischerweise korrekt weiterleiten -- sie modifizieren lediglich Header-Felder und ignorieren den
übrigen IP-Payload -- interpretieren bestimmte Intrusion Detection Systeme und Firewalls die Diskrepanz zwischen
UDP-Length und IP-Length als Angriffsindikator \cite[Sec.~18]{rfc9868}. RFC~9868 adressiert dieses Problem durch
die Option Checksum (OCS).

\section{Zielsetzung}
\label{sec:zielsetzung}

Ziel dieser Arbeit ist die Implementierung einer RFC~9868-konformen Bibliothek für UDP Transport Options in der
Programmiersprache Zig. Die Implementierung fokussiert auf eine Userspace-Lösung mittels Raw Sockets und umfasst
Parser sowie Serializer für TLV-kodierte UDP Options, OCS-Validierung gemäß Section~9 des RFC~9868, Unterstützung
der Must-Support Options (EOL, NOP, APC, FRAG, MDS, MRDS, REQ, RES) sowie Raw-Socket-Programmierung für den Zugriff
auf die Surplus Area.

Die Verifikation der Implementierung erfolgt durch Unit Tests für die einzelnen Komponenten sowie Integrationstests
über das Loopback-Interface. Zusätzlich wird das Verhalten der UDP Options bei der Übertragung über das Internet
untersucht, um die Kompatibilität mit bestehender Netzwerkinfrastruktur zu evaluieren.

Explizit außerhalb des Umfangs dieser Arbeit liegen der REQ/RES-Anwendungsfall für Path MTU Discovery, welcher in
RFC~9869 (DPLPMTUD for UDP Options) definiert ist \cite{rfc9869}, die TIME Option, deren Anwendungsfall im RFC nicht
spezifiziert ist, sowie die Optionen AUTH, UCMP und UENC, die laut Section~11.9 des RFC noch nicht vollständig
spezifiziert sind. Die Implementierung erfolgt ausschließlich im Userspace; Kernel-Module werden nicht entwickelt.

\section{Eingesetzte Technologien}
\label{sec:technologien}

Als zentrale Programmiersprache wurde \textbf{Zig} (Version 0.15.2) gewählt. Zig ist eine relative neue
Systemprogrammiersprache mit manueller Speicherverwaltung ohne versteckte Allokationen -- ein entscheidender Vorteil
für die byte-genaue Konstruktion von Netzwerk-Headern. Die nahtlose C-Interoperabilität ermöglicht die direkte Nutzung
von POSIX-Schnittstellen für Raw Sockets. Zusätzlich bieten integrierte Laufzeitprüfungen einen wirksamen Schutz gegen
Speicherfehler beim Parsen von Netzwerkpaketen.

Für die Analyse des Netzwerkverkehrs und die Verifikation der generierten UDP-Pakete sind \textbf{Wireshark} und
\textbf{tcpdump} unverzichtbar, da sie eine detaillierte Inspektion auf Bitebene erlauben.

Ergänzend wird ein transparenter, KI-gestützter Ansatz verfolgt. Large Language Models (LLMs) wie \textbf{Claude}
(Anthropic) und \textbf{Gemini} (Google) dienen als Assistenzsysteme für Code-Reviews, die Analyse komplexer
RFC-Spezifikationen, \LaTeX{}-Formatierung sowie zur sprachlichen Glättung von Texten.

% =============================================================================
% Kapitel 2: Grundlagen
% =============================================================================

\chapter{Grundlagen}
\label{chap:grundlagen}

\section{Das User Datagram Protocol (UDP)}
\label{sec:udp}

Das User Datagram Protocol (UDP) ist neben dem Transmission Control Protocol (TCP) eines der beiden zentralen
Transportprotokolle der TCP/IP-Protokollfamilie. Im Gegensatz zu TCP verfolgt UDP ein bewusst minimalistisches Design:
Es sendet Datagramme ohne vorherigen Handshake oder Zustandsabgleich mit dem Empfänger, bietet keine Garantien
für Zustellung, Reihenfolge oder Duplikaterkennung und verzichtet
auf jeglichen Flusskontroll- oder Staukontrollmechanismus \cite{rfc768}. Mit der Protokollnummer~17 im IP-Header
registriert, wurde UDP in RFC~768 spezifiziert und trägt die Bezeichnung STD~6 als Internet-Standard. Bemerkenswert
ist, dass der Protokollstandard RFC~768 seit seiner Veröffentlichung im August 1980 nie direkt aktualisiert wurde --
eine Stabilität von über 45~Jahren, die erst mit RFC~9868 durchbrochen wird \cite{rfc9868}.

\subsection{Historische Entwicklung}
\label{subsec:udp-historie}

Die Entstehung von UDP ist untrennbar mit der Aufspaltung des ursprünglichen \emph{Transmission Control Program}
verbunden. In RFC~675 beschrieben Cerf, Dalal und Sunshine im Dezember 1974 ein monolithisches Protokoll, das
sowohl Netzwerk- als auch Transportfunktionen in einem einzigen Protokoll vereinte \cite{rfc675}. Ab 1978 wurde
dieses Konzept schrittweise in separate Schichten aufgeteilt: Das Internet Protocol (IP) übernahm die
Netzwerkfunktionen, während TCP die zuverlässige Transportschicht bildete. Die Erkenntnis, dass nicht alle
Anwendungen die Zuverlässigkeitsgarantien und den damit verbundenen Overhead von TCP benötigen, führte zur
Entwicklung eines leichtgewichtigen Alternativprotokolls -- UDP.

Am 28.~August 1980 veröffentlichte Jon Postel am Information Sciences Institute (ISI) der University of Southern
California RFC~768, die Spezifikation des User Datagram Protocol \cite{rfc768}. Mit nur drei Seiten ist RFC~768
eines der kürzesten RFC-Dokumente überhaupt\footnote{Zum Vergleich: RFC~793 (TCP) umfasst 85~Seiten, RFC~9868
(UDP Options) 51~Seiten.}. Diese Kürze spiegelt das Designziel wider: UDP sollte ein minimalistisches,
transaktionsorientiertes Protokoll sein, das lediglich Portnummern zur Demultiplexierung und eine optionale
Prüfsumme zur Fehlererkennung bereitstellt. Als erste Anwendungsfälle nennt RFC~768 explizit den Internet Name
Server\footnote{Ein Vorläufer des heutigen DNS, das diesen frühen Namensdienst später ablöste.} sowie das
Trivial File Transfer Protocol (TFTP, beschrieben in IEN~133).

Die endgültige Formalisierung des TCP/IP-Stacks erfolgte im September 1981 mit der Veröffentlichung von
RFC~791 (IPv4) \cite{rfc791} und RFC~793 (TCP) \cite{rfc793}. Gemeinsam mit UDP bildeten diese drei Protokolle
die Grundpfeiler der Internet-Protokollsuite. Eine weitere wichtige Ergänzung stellte RFC~1122 dar, in dem Braden
1989 die Implementierungsanforderungen für Internet-Hosts präzisierte \cite{rfc1122}. Dieses Dokument
konkretisierte unter anderem die Anforderungen an UDP-Implementierungen, etwa die Behandlung von Paketen mit
ungültiger Prüfsumme und das Verhalten bei nicht erreichbaren Zielports.

Seit seiner Veröffentlichung 1980 ist der Protokollstandard RFC~768 im Kern unverändert geblieben. Zu den wenigen
ergänzenden Anpassungen zählen die Prüfsummenbehandlung und die Unterstützung von Jumbogrammen: Während die
Prüfsumme bei IPv4 optional ist, schreibt RFC~8200 sie für IPv6 grundsätzlich als verpflichtend vor
\cite[Sec.~8.1]{rfc8200} -- mit der Ausnahme, dass RFC~6936 für bestimmte Tunnelprotokolle einen
Zero-Checksum-Modus erlaubt \cite{rfc6936}. RFC~2675 erweiterte UDP zudem um die Unterstützung von
IPv6-Jumbogrammen, bei denen das UDP-Längenfeld auf null gesetzt wird \cite{rfc2675}. RFC~8085 fasst als
BCP~145 die aktuellen Richtlinien für die Nutzung von UDP zusammen und adressiert insbesondere Fragen der
Staukontrolle bei UDP-basierten Anwendungen \cite{rfc8085}.

Trotz dieser ergänzenden Spezifikationen hat die Bedeutung von UDP in den vergangenen Jahren massiv zugenommen:
Protokolle wie QUIC \cite{rfc9000}, das HTTP/3 als Transportschicht dient \cite{rfc9114}, WebRTC, DNS und
zahlreiche IoT-Anwendungen setzen auf UDP als Basis. Durch die Verbreitung von QUIC und HTTP/3 wird
mittlerweile ein erheblicher und wachsender Anteil des Web-Verkehrs über UDP abgewickelt. Gerade diese
weitgehende Stabilität von RFC~768 bei gleichzeitig wachsender Bedeutung von UDP motiviert RFC~9868 als erste
direkte Erweiterung des Protokollstandards seit über 45~Jahren.

\newpage
\subsection{Protokollstruktur}
\label{subsec:udp-struktur}

Ein UDP-Datagramm besteht aus einem Header mit fester Länge von 8~Byte und den darauf folgenden Nutzdaten.
\rfcref{768} definiert den Header wie folgt \cite{rfc768}:

\begin{verbatim}
   0                             15 16                           31
   +-+-+-+-+-+-+-+-+-+-+-+-+-+-+-+-+-+-+-+-+-+-+-+-+-+-+-+-+-+-+-+-+
   |          Source Port          |       Destination Port        |
   +-+-+-+-+-+-+-+-+-+-+-+-+-+-+-+-+-+-+-+-+-+-+-+-+-+-+-+-+-+-+-+-+
   |            Length             |           Checksum            |
   +-+-+-+-+-+-+-+-+-+-+-+-+-+-+-+-+-+-+-+-+-+-+-+-+-+-+-+-+-+-+-+-+
   |                        data octets ...                        |
   +-+-+-+-+-+-+-+-+-+-+-+-+-+-+-+-+-+-+-+-+-+-+-+-+-+-+-+-+-+-+-+-+
\end{verbatim}

Die vier Felder des Headers haben jeweils eine Breite von 16~Bit:

\begin{itemize}
    \item \code{Source Port} -- Die Portnummer des sendenden Prozesses. Dieses Feld ist optional; wird es nicht
    benötigt, wird es auf null gesetzt \cite{rfc768}. In der Praxis nutzen Anwendungen den \term{Source Port}
    zur Identifikation des Absenders, damit der Empfänger Antworten an den korrekten Port zurücksenden kann.

    \item \code{Destination Port} -- Die Portnummer des Zielprozesses auf dem empfangenden Host. Zusammen mit der
    IP-Zieladresse ermöglicht dieses Feld die \term{Demultiplexierung} eingehender Datagramme an die korrekte
    Anwendung.

    \item \code{Length} -- Die Gesamtlänge des UDP-Datagramms in Byte, bestehend aus Header \emph{und} Nutzdaten.
    Der Minimalwert beträgt~8, was einem Datagramm ohne Nutzdaten entspricht \cite{rfc768}. Da das Feld 16~Bit
    breit ist, ergibt sich eine theoretische Maximalgröße von 65\,535~Byte\footnote{Für IPv6-Jumbogramme wird
    das \code{Length}-Feld auf null gesetzt; die tatsächliche Länge ergibt sich aus dem Jumbo Payload Option Header
    \cite{rfc2675}.}.

    \item \code{Checksum} -- Eine Prüfsumme über einen Pseudo-Header, den UDP-Header und die Nutzdaten. Die
    Berechnung verwendet das \term{Einerkomplement}-Verfahren: Alle 16-Bit-Wörter werden addiert und das
    Einerkomplement der Summe bildet den Prüfsummenwert \cite{rfc1071}. Bei IPv4 ist die Prüfsumme optional --
    wird sie nicht berechnet, wird das Feld auf null gesetzt \cite{rfc768}. Bei IPv6 hingegen ist die Prüfsumme
    verpflichtend, da IPv6 selbst keine Header-Prüfsumme besitzt \cite[Sec.~8.1]{rfc8200}\footnote{\rfcref{6936}
    definiert eine Ausnahme für bestimmte Tunnelprotokolle, bei denen ein Zero-Checksum-Modus erlaubt ist
    \cite{rfc6936}.}.
\end{itemize}

\paragraph{Pseudo-Header.}
Die UDP-Prüfsumme schützt nicht nur die Transportschichtdaten, sondern bezieht über einen \term{Pseudo-Header}
auch Informationen der Vermittlungsschicht ein. Dieses Konzept stellt sicher, dass ein Datagramm, das durch einen
IP-Fehler an den falschen Host oder das falsche Protokoll zugestellt wird, erkannt und verworfen werden kann.
Für IPv4 definiert \rfcref{768} den folgenden Pseudo-Header \cite{rfc768}:

\begin{center}
\begin{BVerbatim}
 0      7 8     15 16    23 24    31
+--------+--------+--------+--------+
|          Source Address           |
+--------+--------+--------+--------+
|        Destination Address        |
+--------+--------+--------+--------+
|  Zero  |Protocol|   UDP Length    |
+--------+--------+--------+--------+
\end{BVerbatim}
\end{center}

Bei IPv6 erweitert sich der Pseudo-Header auf die 128-Bit-Adressen. Anstelle des 8-Bit-Protokollfeldes verwendet
IPv6 ein 8-Bit-\code{Next Header}-Feld (Wert~17 für UDP) sowie ein 32-Bit-Feld für die Paketlänge der
Transportschicht. Die verbleibenden 24~Bit werden mit Nullen aufgefüllt \cite[Sec.~8.1]{rfc8200}:

\begin{center}
\begin{BVerbatim}
+-+-+-+-+-+-+-+-+-+-+-+-+-+-+-+-+-+-+-+-+-+-+-+-+-+-+-+-+-+-+-+-+
|                                                               |
+                   Source Address (128 Bit)                    +
|                                                               |
+-+-+-+-+-+-+-+-+-+-+-+-+-+-+-+-+-+-+-+-+-+-+-+-+-+-+-+-+-+-+-+-+
|                                                               |
+                 Destination Address (128 Bit)                 +
|                                                               |
+-+-+-+-+-+-+-+-+-+-+-+-+-+-+-+-+-+-+-+-+-+-+-+-+-+-+-+-+-+-+-+-+
|                   Upper-Layer Packet Length                   |
+-+-+-+-+-+-+-+-+-+-+-+-+-+-+-+-+-+-+-+-+-+-+-+-+-+-+-+-+-+-+-+-+
|                      Zero                     |  Next Header  |
+-+-+-+-+-+-+-+-+-+-+-+-+-+-+-+-+-+-+-+-+-+-+-+-+-+-+-+-+-+-+-+-+
\end{BVerbatim}
\end{center}

\newpage
\subsection{Eigenschaften}
\label{subsec:udp-eigenschaften}

Die in \secref{subsec:udp-struktur} beschriebene Header-Struktur verdeutlicht bereits das minimalistische Design
von UDP. Aus diesem Designziel leiten sich eine Reihe funktionaler Eigenschaften ab, die maßgeblich beeinflussen,
für welche Anwendungsfälle das Protokoll geeignet ist.

\paragraph{Nachrichtenorientierung.}
UDP ist ein \term{nachrichtenorientiertes} Protokoll: Jede Sendeoperation erzeugt genau ein Datagramm, und der
Empfänger erhält jede Nachricht als atomare Einheit. \rfcref{768} beschreibt UDP als Verfahren, mit dem Programme Nachrichten an andere Programme senden können
(\zitat{a procedure to send [...] messages to other programs}) \cite{rfc768}.

\paragraph{Zustandslosigkeit.}
UDP verwaltet keinen protokollspezifischen Zustand für einzelne Kommunikationsbeziehungen und hält keinerlei
Datenstrukturen wie Sequenznummern, Fenstergrößen oder Retransmission-Timer vor \cite[Sec.~6]{rfc9868}.

\paragraph{Minimale Latenz.}
Da UDP verbindungslos arbeitet, entfällt jeglicher Handshake vor der Datenübertragung. Die erste Nachricht kann
unmittelbar gesendet werden, ohne dass ein vorheriger Verbindungsaufbau abgewartet werden muss
\cite[Sec.~1]{rfc8085}. Ebenso existiert kein Verbindungsabbau, der
zusätzliche Latenz verursachen könnte.

\paragraph{Unabhängigkeit der Datagramme.}
Jedes UDP-Datagramm ist eine eigenständige, von allen anderen Datagrammen unabhängige Einheit. Der Verlust eines
Datagramms hat daher keine Auswirkung auf die Verarbeitung nachfolgender Datagramme. Diese Eigenschaft vermeidet
das sogenannte \term{Head-of-Line-Blocking} (HoL-Blocking), bei dem verlorene Pakete die Auslieferung
nachfolgender Daten verzögern. Diese Eigenschaft von UDP war eine der Motivationen für die Entwicklung von QUIC:
QUIC nutzt UDP
als Transportschicht und realisiert auf Anwendungsebene ein Stream-Multiplexing, bei dem der Verlust eines Pakets
nur den betroffenen Stream blockiert, nicht jedoch andere parallele Streams \cite[Sec.~1]{rfc9000}.

\paragraph{Multicast und Broadcast.}
UDP ist verbindungslos und wird häufig als Transport für IP-Multicast und (unter IPv4) IP-Broadcast verwendet.
Ein UDP-Datagramm kann an eine Multicast-Gruppe oder eine Broadcast-Adresse gesendet und von mehreren Empfängern
empfangen werden, sofern Netzwerk und Hosts entsprechend konfiguriert sind. Dabei gelten die üblichen
Eigenschaften von UDP: keine Garantie für Zustellung, Reihenfolge oder Duplikaterkennung. Diese Kombination
eignet sich besonders für lokale Mechanismen wie Service Discovery sowie für bestimmte
Echtzeitanwendungen in kontrollierten Netzen.

Diese Eigenschaften machen UDP zu einem attraktiven Transportprotokoll für latenzempfindliche, verlusttolerante
oder gruppenbasierte Anwendungen. Allerdings fehlt UDP jede Form der protokolleigenen Erweiterbarkeit. Genau
diese Lücke adressiert \rfcnine{} durch die Nutzung der
\surplusarea{} (siehe \secref{subsec:surplus-area}), um Optionen zu transportieren, ohne den bestehenden
UDP-Header zu verändern.

\section{Internet Protocol (IP)}
\label{sec:ip}

\subsection{IPv4}
\label{subsec:ipv4}

\subsection{IPv6}
\label{subsec:ipv6}

\section{Middleboxes und Protokoll-Ossifikation}
\label{sec:middlebox}

\subsection{Internet-Ossifikation}
\label{subsec:ossifikation}

\subsection{Auswirkungen auf Protokollerweiterungen}
\label{subsec:ossifikation-auswirkungen}

\section{RFC 9868: Transport Options for UDP}
\label{sec:rfc9868}

\subsection{Überblick}
\label{subsec:rfc9868-overview}

\subsection{Der Surplus Area}
\label{subsec:surplus-area}

\subsection{Options-Struktur}
\label{subsec:options-struktur}

\subsection{Definierte Options-Typen}
\label{subsec:options-typen}

\subsection{SAFE und UNSAFE Options}
\label{subsec:safe-unsafe}

\section{Checksum Offloading}
\label{sec:checksum-offloading}

% =============================================================================
% Kapitel 3: Analyse
% =============================================================================

\chapter{Analyse}
\label{chap:analyse}

\section{Anforderungsanalyse}
\label{sec:anforderungen}

\subsection{Funktionale Anforderungen}
\label{subsec:funktionale-anforderungen}

\subsection{Nicht-funktionale Anforderungen}
\label{subsec:nicht-funktionale-anforderungen}

\subsection{Speichereffizienz und Zero-Copy}
\label{subsec:zero-copy}

\section{Existierende Implementierungen}
\label{sec:existierende-implementierungen}

\subsection{Linux Kernel}
\label{subsec:linux-kernel}

\subsection{Userspace-Implementierungen}
\label{subsec:userspace}

\section{Herausforderungen}
\label{sec:herausforderungen}

\subsection{Zugriff auf den Surplus Area}
\label{subsec:surplus-zugriff}

\subsection{NAT-Traversal}
\label{subsec:nat-traversal}

\subsection{Interoperabilität}
\label{subsec:interoperabilitaet}

\subsection{Alternative Implementierungsansätze}
\label{subsec:alternativen}
% eBPF/XDP, Kernel Module vs. Raw Sockets

\section{Technologieauswahl}
\label{sec:technologieauswahl}

\subsection{Begründung für Zig}
\label{subsec:begruendung-zig}

\subsection{Architekturentscheidungen}
\label{subsec:architekturentscheidungen}

% =============================================================================
% Kapitel 4: Konzeption
% =============================================================================

\chapter{Konzeption}
\label{chap:konzeption}

\section{Architekturübersicht}
\label{sec:architektur}

\section{Datenstrukturen}
\label{sec:datenstrukturen}

\subsection{IP-Header}
\label{subsec:ip-header}

\subsection{UDP-Header}
\label{subsec:udp-header}

\subsection{UDP Options}
\label{subsec:udp-options-struct}

\section{Options-Verarbeitung}
\label{sec:options-verarbeitung}

\subsection{Parsing-Algorithmus}
\label{subsec:parsing}

\subsection{Serialisierungs-Algorithmus}
\label{subsec:serialisierung}

\section{Checksum-Berechnung}
\label{sec:checksum}

\subsection{Option Checksum (OCS)}
\label{subsec:ocs-berechnung}

\section{Fragmentierung}
\label{sec:fragmentierung}

\subsection{FRAG Option-Konzept}
\label{subsec:frag-konzept}

\subsection{Reassembly-Strategie}
\label{subsec:reassembly}

\section{API-Design}
\label{sec:api-design}

\subsection{High-Level API}
\label{subsec:high-level-api}

\subsection{Low-Level API}
\label{subsec:low-level-api}

\section{Fehlerbehandlung}
\label{sec:fehlerbehandlung}

\section{State Management}
\label{sec:state-management}

\subsection{Zustandsspeicherung für Fragmentierung}
\label{subsec:fragment-state}

\subsection{Timeout-Handling und Garbage Collection}
\label{subsec:timeout-gc}

\section{Sicherheitsüberlegungen}
\label{sec:sicherheit}

\subsection{Denial-of-Service-Schutz}
\label{subsec:dos-schutz}

\subsection{Behandlung malformierter Options}
\label{subsec:malformed-options}

% =============================================================================
% Kapitel 5: Realisierung
% =============================================================================

\chapter{Realisierung}
\label{chap:realisierung}

\section{Die Programmiersprache Zig}
\label{sec:zig}

\subsection{Überblick}
\label{subsec:zig-overview}

\subsection{Relevante Sprachfeatures}
\label{subsec:zig-features}

\subsection{Raw Sockets in Zig}
\label{subsec:zig-sockets}

\section{Projektstruktur}
\label{sec:projektstruktur}

\section{Raw Socket-Implementierung}
\label{sec:raw-socket}

\subsection{Socket-Erstellung}
\label{subsec:socket-erstellung}

\subsection{Paketempfang}
\label{subsec:paketempfang}

\subsection{Plattform-Spezifika}
\label{subsec:plattform}
% Linux vs. macOS/Darwin Unterschiede

\section{Options-Parser}
\label{sec:options-parser}

\subsection{Hauptparser}
\label{subsec:hauptparser}

\subsection{Spezifische Parser}
\label{subsec:spezifische-parser}

\section{Options-Serialisierer}
\label{sec:options-serialisierer}

\subsection{Serialisierung}
\label{subsec:serialisierung-impl}

\section{OCS-Implementierung}
\label{sec:ocs-implementierung}

\subsection{Checksum-Berechnung}
\label{subsec:checksum-impl}

\subsection{OCS-Validierung}
\label{subsec:ocs-validierung}

\section{Fragmentierung}
\label{sec:frag-implementierung}

\subsection{Fragment-Cache}
\label{subsec:fragment-cache-impl}

\section{Build-Konfiguration}
\label{sec:build}

% =============================================================================
% Kapitel 6: Evaluation
% =============================================================================

\chapter{Evaluation}
\label{chap:evaluation}

\section{Testkonzept}
\label{sec:testkonzept}

\section{Unit-Tests}
\label{sec:unit-tests}

\subsection{Parser-Tests}
\label{subsec:parser-tests}

\subsection{OCS-Tests}
\label{subsec:ocs-tests}

\subsection{Serialisierer-Tests}
\label{subsec:serialisierer-tests}

\section{Integrationstests}
\label{sec:integrationstests}

\subsection{End-to-End-Tests}
\label{subsec:e2e-tests}

\subsection{Fragmentierungs-Tests}
\label{subsec:frag-tests}

\section{Performanz-Evaluation}
\label{sec:performanz}

\subsection{Messaufbau}
\label{subsec:messaufbau}

\subsection{Parsing-Performanz}
\label{subsec:parsing-performanz}

\subsection{Vergleich mit Standard-UDP}
\label{subsec:vergleich-udp}

\subsection{Baseline-Definition}
\label{subsec:baseline}

\subsection{CPU-Overhead-Analyse}
\label{subsec:cpu-overhead}

\section{Interoperabilitätstests}
\label{sec:interop}

\subsection{Testpartner}
\label{subsec:testpartner}

\subsection{Testergebnisse}
\label{subsec:interop-ergebnisse}

\section{Erfüllung der Anforderungen}
\label{sec:anforderungserfuellung}

% =============================================================================
% Kapitel 7: Fazit
% =============================================================================

\chapter{Fazit}
\label{chap:fazit}

\section{Zusammenfassung}
\label{sec:zusammenfassung}

\section{Erkenntnisse}
\label{sec:erkenntnisse}

\subsection{Technische Erkenntnisse}
\label{subsec:tech-erkenntnisse}

\subsection{Methodische Erkenntnisse}
\label{subsec:method-erkenntnisse}

\section{Limitationen}
\label{sec:limitationen}

\section{Ausblick}
\label{sec:ausblick}

\subsection{Kurzfristige Erweiterungen}
\label{subsec:kurzfristig}

\subsection{Langfristige Perspektiven}
\label{subsec:langfristig}

\subsection{Forschungsfragen}
\label{subsec:forschungsfragen}

\section{Schlusswort}
\label{sec:schlusswort}


% -----------------------------------------------------------------------------
% BACKMATTER
% -----------------------------------------------------------------------------

% Literaturverzeichnis
\printbibliography[heading=bibintoc, title=Literaturverzeichnis]

% --- EIDESSTATTLICHE ERKLÄRUNG (ehemals frontmatter/declaration.tex) ---
\chapter*{Eidesstattliche Erklärung}
\addcontentsline{toc}{chapter}{Eidesstattliche Erklärung}

Ich versichere, dass ich die vorliegende Arbeit selbstständig verfasst und keine anderen als die angegebenen
Quellen und Hilfsmittel benutzt habe.

Alle Stellen der Arbeit, die wörtlich oder sinngemäß aus Veröffentlichungen oder aus anderweitigen fremden
Quellen entnommen wurden, sind als solche kenntlich gemacht.

Die Arbeit wurde in gleicher oder ähnlicher Form noch keiner Prüfungsbehörde vorgelegt.

\vspace{3cm}

\noindent
\begin{tabular}{p{6cm}p{1cm}p{6cm}}
    \dotfill & & \dotfill \\
    Ort, Datum & & Unterschrift \\
\end{tabular}

\end{document}
