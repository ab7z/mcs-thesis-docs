% =============================================================================
% Kapitel 4: Konzeption
% =============================================================================

\chapter{Konzeption}
\label{chap:konzeption}

\section{Architekturübersicht}
\label{sec:architektur}

\section{Datenstrukturen}
\label{sec:datenstrukturen}

\subsection{IP-Header}
\label{subsec:ip-header}

\subsection{UDP-Header}
\label{subsec:udp-header}

\subsection{UDP Options}
\label{subsec:udp-options-struct}

\section{Options-Verarbeitung}
\label{sec:options-verarbeitung}

\subsection{Parsing-Algorithmus}
\label{subsec:parsing}

\subsection{Serialisierungs-Algorithmus}
\label{subsec:serialisierung}

\section{Checksum-Berechnung}
\label{sec:checksum}

\subsection{Option Checksum (OCS)}
\label{subsec:ocs-berechnung}

\section{Fragmentierung}
\label{sec:fragmentierung}

\subsection{FRAG Option-Konzept}
\label{subsec:frag-konzept}

\subsection{Reassembly-Strategie}
\label{subsec:reassembly}

\section{API-Design}
\label{sec:api-design}

\subsection{High-Level API}
\label{subsec:high-level-api}

\subsection{Low-Level API}
\label{subsec:low-level-api}

\section{Fehlerbehandlung}
\label{sec:fehlerbehandlung}

\section{State Management}
\label{sec:state-management}

\subsection{Zustandsspeicherung für Fragmentierung}
\label{subsec:fragment-state}

\subsection{Timeout-Handling und Garbage Collection}
\label{subsec:timeout-gc}

\section{Sicherheitsüberlegungen}
\label{sec:sicherheit}

\subsection{Denial-of-Service-Schutz}
\label{subsec:dos-schutz}

\subsection{Behandlung malformierter Options}
\label{subsec:malformed-options}
