% =============================================================================
% Kapitel 3: Analyse
% =============================================================================

\chapter{Analyse}
\label{chap:analyse}

\section{Anforderungsanalyse}
\label{sec:anforderungen}

\subsection{Funktionale Anforderungen}
\label{subsec:funktionale-anforderungen}

\subsection{Nicht-funktionale Anforderungen}
\label{subsec:nicht-funktionale-anforderungen}

\subsection{Speichereffizienz und Zero-Copy}
\label{subsec:zero-copy}

\section{Existierende Implementierungen}
\label{sec:existierende-implementierungen}

\subsection{Linux Kernel}
\label{subsec:linux-kernel}

\subsection{Userspace-Implementierungen}
\label{subsec:userspace}

\section{Herausforderungen}
\label{sec:herausforderungen}

\subsection{Zugriff auf den Surplus Area}
\label{subsec:surplus-zugriff}

\subsection{NAT-Traversal}
\label{subsec:nat-traversal}

\subsection{Interoperabilität}
\label{subsec:interoperabilitaet}

\subsection{Alternative Implementierungsansätze}
\label{subsec:alternativen}
% eBPF/XDP, Kernel Module vs. Raw Sockets

\section{Technologieauswahl}
\label{sec:technologieauswahl}

\subsection{Begründung für Zig}
\label{subsec:begruendung-zig}

\subsection{Architekturentscheidungen}
\label{subsec:architekturentscheidungen}
