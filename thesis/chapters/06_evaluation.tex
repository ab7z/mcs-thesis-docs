% =============================================================================
% Kapitel 6: Evaluation
% =============================================================================

\chapter{Evaluation}
\label{chap:evaluation}

\section{Testkonzept}
\label{sec:testkonzept}

\section{Unit-Tests}
\label{sec:unit-tests}

\subsection{Parser-Tests}
\label{subsec:parser-tests}

\subsection{OCS-Tests}
\label{subsec:ocs-tests}

\subsection{Serialisierer-Tests}
\label{subsec:serialisierer-tests}

\section{Integrationstests}
\label{sec:integrationstests}

\subsection{End-to-End-Tests}
\label{subsec:e2e-tests}

\subsection{Fragmentierungs-Tests}
\label{subsec:frag-tests}

\section{Performanz-Evaluation}
\label{sec:performanz}

\subsection{Messaufbau}
\label{subsec:messaufbau}

\subsection{Parsing-Performanz}
\label{subsec:parsing-performanz}

\subsection{Vergleich mit Standard-UDP}
\label{subsec:vergleich-udp}

\subsection{Baseline-Definition}
\label{subsec:baseline}

\subsection{CPU-Overhead-Analyse}
\label{subsec:cpu-overhead}

\section{Interoperabilitätstests}
\label{sec:interop}

\subsection{Testpartner}
\label{subsec:testpartner}

\subsection{Testergebnisse}
\label{subsec:interop-ergebnisse}

\section{Erfüllung der Anforderungen}
\label{sec:anforderungserfuellung}
