% =============================================================================
% Kapitel 2: Grundlagen
% =============================================================================

\chapter{Grundlagen}
\label{chap:grundlagen}

\section{Das User Datagram Protocol (UDP)}
\label{sec:udp}

\subsection{Historische Entwicklung}
\label{subsec:udp-historie}

\subsection{Protokollstruktur}
\label{subsec:udp-struktur}

\subsection{Eigenschaften}
\label{subsec:udp-eigenschaften}

\section{Internet Protocol (IP)}
\label{sec:ip}

\subsection{IPv4}
\label{subsec:ipv4}

\subsection{IPv6}
\label{subsec:ipv6}

\section{Middleboxes und Protokoll-Ossifikation}
\label{sec:middlebox}

\subsection{Internet-Ossifikation}
\label{subsec:ossifikation}

\subsection{Auswirkungen auf Protokollerweiterungen}
\label{subsec:ossifikation-auswirkungen}

\section{RFC 9868: Transport Options for UDP}
\label{sec:rfc9868}

\subsection{Überblick}
\label{subsec:rfc9868-overview}

\subsection{Der Surplus Area}
\label{subsec:surplus-area}

\subsection{Options-Struktur}
\label{subsec:options-struktur}

\subsection{Definierte Options-Typen}
\label{subsec:options-typen}

\subsection{SAFE und UNSAFE Options}
\label{subsec:safe-unsafe}

\section{Checksum Offloading}
\label{sec:checksum-offloading}
